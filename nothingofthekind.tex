\documentclass[
%french,
%draft,
letterpaper,
% paper=A4,
% version=last,
% NF,
12pt
% ,
% toc=bibnumbered
]{article}
\usepackage{scrextend}
\changefontsizes{14pt}
\sloppy
\usepackage{adjustbox}
\usepackage[english, french]{babel}
\usepackage[%-----------------------------------------------------------
bibencoding=auto
,backend=biber
,sorting=none
%,autolang%error: no value specified for autolang.
]{biblatex}
%\selectbiblanguage{french}
\DefineBibliographyStrings{french}
{ in = {dans}, }

\begin{filecontents}{\jobname.bib}
@book{alarech-2,
        author = {Marcel Proust},
        title = {À l'ombre des jeunes filles en fleur},
        year = {1919},
        publisher = {Gallimard}
}
\end{filecontents}
\addbibresource{\jobname.bib}

\usepackage{etoolbox}
\def\caviardeFlag{}

\usepackage[autostyle]{csquotes}
\usepackage[useregional]{datetime2}
\usepackage{datetime2}
%\DTMsetdatestyle{ddmmyyyy}
%\DTMsetup{datesep={/}}
\usepackage[inline]{enumitem}
\usepackage[T1]{fontenc}

%\usepackage{libertine}
%\usepackage{graphicx}
\usepackage[svgnames]{xcolor}
\usepackage{framed}

\usepackage{geometry}

\usepackage{lipsum} % TEMP

\newcommand*\openquote{\makebox(25,-22){\scalebox{5}{``}}}
\newcommand*\closequote{\makebox(25,-22){\scalebox{5}{''}}}

\usepackage[many]{tcolorbox}
\usepackage{titlecaps}
\usepackage{xparse}

\usepackage{comment}
\usepackage[inline]{enumitem}
\setlist[itemize]{itemjoin=\hspace*{\fill},itemjoin*=\hspace*{\fill}}
\usepackage{keyfloat}
\usepackage{lastpage}
\usepackage{textcomp}
\usepackage{xwatermark}
% <----------------------------------------------------------------------
% https://tex.stackexchange.com/questions/54946/how-to-break-long-url-in-an-item#254734
\usepackage{hyperref}
\def\urlbreaks{\do\/\do-}
% ---------------------------------------------------------------------->

\usepackage{fancyhdr}%-------------------------------------------------
\fancyhf{}
%% \renewcommand{\headrulewidth}{0pt}
\fancyhead[c]{Rien au contraire}
\fancyfoot[c]{\thepage/\pageref{LastPage}}

%-----------------------------------------------------------------------

\newenvironment{docabstract}[1]%https://latex.org/forum/viewtopic.php?t=12156
{\renewcommand{\abstractname}{#1}\begin{abstract}}
  {\end{abstract}}

\selectlanguage{french}

\title{Rien au contraire\\ (nothing of the kind)}

\date{\today}

\begin{document}
\pagestyle{fancy}
\selectlanguage{french}

\maketitle

\begin{docabstract}{Résumé}
Sous la forme d'un Q\&R, commentaire de texte du dialogue de \textquote{À l'ombre des jeunes filles en fleur} où il est dit \textquote{Rien au contraire}\cite[Volume II]{alarech-2}. Survenu dans le cadre d'un groupe informel de francophiles\footnote{Celui associé à \url{https://www.facebook.com/floridefrancaise/}}.
\end{docabstract}

   \selectlanguage{english}
   \begin{docabstract}{Abstract}
By way of a Q\&A, text commentary of the dialogue from \textquote{In the shadow of young girls in flower} that contains: \textquote{Nothing of the kind}\footnote{As translated by James Grieve}. Arose as part of an informal group of Francophiles.
  \end{docabstract}

   \selectlanguage{french}

\ifcsundef{caviardeFlag}
{\StopCensoring}
{}

\newlist{quest}{enumerate}{1}
\setlist[quest]{resume,leftmargin=*,label=\emph{\arabic*)}}

%\colorlet{shadecolor}{Azure}
%\colorlet{shadecolor}{Ivory}
%\colorlet{shadecolor}{Beige}
\colorlet{shadecolor}{BlanchedAlmond}
\begin{snugshade}\openquote
A M\up{me} de Villeparisis qui le priait de décrire pour ma grand’mère un château où avait séjourné M\up{me} de Sévigné, ajoutant qu’elle voyait un peu de littérature dans ce désespoir d’être séparée de cette ennuyeuse M\up{me} de Grignan:\\
--- \textquote{Rien au contraire, répondit-il, ne me semble plus vrai. C’était du reste une époque où ces sentiments-là étaient bien compris. L’habitant du Monomopata de Lafontaine, courant chez son ami qui lui est apparu un peu triste pendant son sommeil, le pigeon trouvant que le plus grand des maux est l’absence de l’autre pigeon, vous semblent peut-être, ma tante, aussi exagérés que M\up{me} de Sévigné ne pouvant pas attendre le moment où elle sera seule avec sa fille. C’est si beau ce qu’elle dit quand elle la quitte: cette séparation me fait une douleur à l’âme que je sens comme un mal du corps. Dans l’absence on est libéral des heures. On avance dans un temps auquel on aspire.} %Ma grand’mère était ravie d’entendre parler de ces Lettres, exactement de la façon qu’elle eût fait. Elle s’étonnait qu’un homme pût les comprendre si bien.
\closequote
\end{snugshade}

Ce passage rapporte un dialogue:%, et le tout début d'une réflexion sur celui-ci:
\begin{quest}
  \item Quelles sont les parties en présence, et leurs liens de parenté?
\end{quest}  
Le dialogue porte sur deux personnes:
\begin{quest}
\item Lesquelles, nommément, et quel est leur lien de parenté?
\item À quoi sait-on qu'elles apartiennent à un passé révolu?
\end{quest}
À l'évocation de \textquote{un peu de littérature} par M\up{me} de Villeparisis, son interlocuteur répond: \textquote{Rien ne me semble plus vrai}
\begin{quest}
\item À quoi se rapporte \textquote{vrai} ?
\item Quel adjectif pour exprimer \textquote{un peu de littérature} ?
\item Quel est le thème de Monopata de La Fontaine ?
\end{quest}

\textquote{M\up{me} de Sévigné ne pouvant pas attendre le moment où elle sera seule avec sa fille}
\begin{quest}
\item Quelle pourrait être la cause de leur séparation, et quel indice le confirme?
\end{quest}  

\textquote{Dans l'absence, on est libéral des heures}:
\begin{quest}
\item Le temps passe vite, lentement, ou quoi d'autre ?
\item Dans quel contexte serait-ce faux ?
\end{quest}  

Juste à la suite du premier passage:
\begin{snugshade}\openquote
Ma grand’mère était ravie d’entendre parler de ces Lettres, exactement de la façon qu’elle eût fait. Elle s’étonnait qu’un homme pût les comprendre si bien.\closequote
\end{snugshade}

\begin{quest}
\item Quel trait de personnalité la grand’mère du narrateur a t-elle en commun avec le neveu de M\up{me} de Villeparisis ?
\end{quest}  
%Ce passage est extrait de \textquote{À l'ombre des jeunes filles en fleur}:
%\begin{quest}
%\item comment exprimer cela avec \textquote{dans la lumière} et \textquote{dans l'ombre} ?
%\end{quest}  


%\PhantomSection*{subsection}{Bibliographie}{bib}
%\addcontentsline{toc}{section}{Bibliographie}
\printbibliography[heading=subbibliography]

\vspace*{\fill}

\keyfig[H]
{lw=0.2,cstar={}}
{/home/er/Documents/write/edu/topic/hum/french/cc-icon.png}


\end{document}