\documentclass[
  %a4paper,
  12pt]
{scrbook}%reprt} 
\RequirePackage[german, french]{babel} %French is the default
\RequirePackage[T1]{fontenc}
\RequirePackage{scrlayer-scrpage}
\RequirePackage{comment}
\RequirePackage[short, nodayofweek]{datetime}
%\RequirePackage{sidenotes}%{marginnote} 
%\RequirePackage[useregional]{datetime2}
%\RequirePackage{paracol}%textcol
\RequirePackage{hyperref}
%\RequirePackage[usenames, dvipsnames]{color} % textcol
%\RequirePackage{yfonts} % textfrak

\input{lecondall-macro.tex}
%\title{La le\c con d'Allemand syst\'ematique -- Original}
 
\begin{document}

%\part{Premi\`ere partie}

% CH 1 - MANN �BER BORD
%
\chapter{Mann \"uber Bord}\label{ch:mann}

\section*{\ldaTXT}
\section*{\ldaLEX}
	\subsection*{\ldaLEXa}
	\subsection*{\ldaLEXb}
\section*{\ldaGRAM}
\section*{\ldaEXs}
\section*{\ldaLEXc}
\section*{\ldaTHEME}
\section*{\ldaDISK}


% CH 2 - DIE WELT DER SICHERHEIT
\chapter{Die Welt der Sicherheit}\label{ch:sicherheit}

\section*{\ldaTXT}
\section*{\ldaLEX}
	\subsection*{\ldaLEXa}
	\subsection*{\ldaLEXb}
\section*{\ldaGRAM}
\section*{\ldaEXs}
\section*{\ldaLEXc}
\section*{\ldaTHEME}
\section*{\ldaDISK}

% CH 3 - LEBEN MIT CHEMIE

\chapter{Leben mit Chemie}\label{ch:chemie}

\section*{\ldaTXT}
\section*{\ldaLEX}
	\subsection*{\ldaLEXa}
	\subsection*{\ldaLEXb}
\section*{\ldaGRAM}
	\subsection*{Expression de l'hypoth\`ese}\label{chemie:hyp}
	\subsection*{Le verbe wissen}\label{chemie:wissen}
	\subsection*{Sondern}\label{chemie:sondern}
	\subsection*{La pr\'eposition `aus'}\label{chemie:aus}
\section*{\ldaEXs}
\section*{\ldaLEXc}
	\subsection*{Chemie im Alltag}\label{chemie:alltag}
	\subsection*{Deutsche Fl\"usse une Nebenfl\"usse}\label{chemie:flusse}
\section*{\ldaTHEME}
\section*{\ldaDISK}

% CH 4 - HABEN ODER SEIN

\chapter{Sein oder Haben}\label{ch:seinoder}

\section*{\ldaTXT}
\section*{\ldaLEX}
	\subsection*{\ldaLEXa}
	\subsection*{\ldaLEXb}
\section*{\ldaGRAM}
	\subsection*{Le relatif was}\label{sein:was}
	\subsection*{bleiben, blieb, geblieben}\label{sein:bleiben}
	\subsection*{Le verbe gelingen}\label{sein:gelingen}
	\subsection*{aussen, \"ausserlich, draussen, ausser, ausserhalb, etc.}\label{sein:aussen}
	\subsection*{machen + adjectif}\label{sein:machen}	
\section*{\ldaEXs}
\section*{\ldaLEXc}
	\subsection*{Die Konsumgesellschaft}\label{sein:konsum}
	\subsection*{Der Markt und die Werbung}\label{sein:markt}
\section*{\ldaTHEME}
\section*{\ldaDISK}

% CH 5 - DER SPIELVERDERBER UND DIE REAKTION�REN KR�FTE

\chapter{Der Spielverderber und die reaktion\"aren Kr\"afte}\label{ch:spiel}

\section*{\ldaTXT}
\section*{\ldaLEX}
	\subsection*{\ldaLEXa}
	\subsection*{\ldaLEXb}
\section*{\ldaGRAM}
\subsection*{Infinitif substantiv\'e}
	\label{spiel:infsubs}
\subsection*{schw\"oren, schwor, geschworent}
	\label{spiel:schw}
\subsection*{wann \& wenn}
	\label{spiel:wann}
\subsection*{wie+adjectif ou verbe}
	\label{spiel:wie}
\subsection*{Expression de la simultan\'eit\'e}
	\label{spiel:simult}
\section*{\ldaEXs}
\section*{\ldaLEXc}
	\subsection*{Die Gesellschaft im Wandel}\label{spiel:wandel}
\section*{\ldaTHEME}\label{spiel:theme}
\section*{\ldaDISK}\label{spiel:disk}

% CH 6 - WALRECHT AUCH F�R AUSL�NDER (FRANKFURTER RUNDSCHAU)

\chapter{Walrech auch f\"ur Ausl\"ander}\label{ch:walrecht}

\section*{\ldaTXT}
\section*{\ldaLEX}
	\subsection*{\ldaLEXa}
	\subsection*{\ldaLEXb}
\section*{\ldaGRAM}
\subsection*{Infinitif substantiv\'e}
	\label{spiel:infsubs}
\subsection*{schw\"oren, schwor, geschworent}
	\label{spiel:schw}
\subsection*{wann \& wenn}
	\label{spiel:wann}
\subsection*{wie+adjectif ou verbe}
	\label{spiel:wie}
\subsection*{Expression de la simultan\'eit\'e}
	\label{spiel:simult}
\section*{\ldaEXs}
\section*{\ldaLEXc}
\section*{\ldaTHEME}\label{spiel:theme}
\section*{\ldaDISK}\label{spiel:disk}

% CH 7 - VERSICHERUNGMORAL 

\chapter{Versicherungsmoral}\label{ch:versicherung}

\section*{\ldaTXT}
\section*{\ldaLEX}
	\subsection*{\ldaLEXa}
	\subsection*{\ldaLEXb}
\section*{\ldaGRAM}
\subsection*{Infinitif substantiv\'e}
	\label{spiel:infsubs}
\subsection*{schw\"oren, schwor, geschworent}
	\label{spiel:schw}
\subsection*{wann \& wenn}
	\label{spiel:wann}
\subsection*{wie+adjectif ou verbe}
	\label{spiel:wie}
\subsection*{Expression de la simultan\'eit\'e}
	\label{spiel:simult}
\section*{\ldaEXs}
\section*{\ldaLEXc}
\section*{\ldaTHEME}\label{spiel:theme}
\section*{\ldaDISK}\label{spiel:disk}

% CH 8 - ZWEI WELTEN

\chapter{Zwei Welten}\label{ch:zweiwelten}

\section*{\ldaTXT}
\section*{\ldaLEX}
	\subsection*{\ldaLEXa}
	\subsection*{\ldaLEXb}
\section*{\ldaGRAM}
\subsection*{Es gibt (+accusatif)}
	\label{zweiwelten:esgibt}
\subsection*{Man}
	\label{zweiwelten:man}
\subsection*{Heben, hob, gehoben}
	\label{zweiwelten:heben}
\subsection*{an\ldots vorbei -- pr\'epositions compos\'ees disjointes}
	\label{zweiwelten:vorbei}
\subsection*{Pr\'everbes ins\'eparables}
	\label{zweiwelten:preverbes}
\section*{\ldaEXs}
\section*{\ldaLEXc}
\section*{\ldaTHEME}\label{spiel:theme}
\section*{\ldaDISK}\label{spiel:disk}

% CH 9 - BELOHNUNG F�R FAHNUNGSHILFE

\chapter{Belohnung f\"ur fahnungshilfe}\label{ch:belohnung}

\section*{\ldaTXT}
\section*{\ldaLEX}
	\subsection*{\ldaLEXa}
	\subsection*{\ldaLEXb}
\section*{\ldaGRAM}
\subsection*{Pronom relatif au g\'enitif: dessen \& deren}
	\label{spiel:pronom}
\subsection*{Erst \& nur}
	\label{zweiwelten:erst}
\subsection*{Nach, nachdem, nachder}
	\label{zweiwelten:nach}
\subsection*{Subordonn\'ees condtionnelles}
	\label{zweiwelten:subcond}
\section*{\ldaEXs}
\section*{\ldaLEXc}
\section*{\ldaTHEME}\label{spiel:theme}
\section*{\ldaDISK}\label{spiel:disk}

% CH 10 - PATRIOTISCHE BEGEISTERUNG

\chapter{Patriotische Begeisterung}\label{ch:patriot}

\section*{\ldaTXT}
\section*{\ldaLEX}
	\subsection*{\ldaLEXa}
	\subsection*{\ldaLEXb}
\section*{\ldaGRAM}
\subsection*{L'apposition}
	\label{patriot:apposition}
\subsection*{Einander: pronom personnel}
	\label{patriot:einander}
\subsection*{Lang: adjectif postpos\'e invariable}
	\label{patriot:lang}
\subsection*{Beginnen, begann, begonnen}
	\label{patriot:beginnen}
\subsection*{La relative}
	\label{patriot:relative}
\subsection*{Immer, immer noch, \ldots, wieder, \ldots, zu, schon immer}
	\label{patriot:immer}
\section*{\ldaEXs}
\section*{\ldaLEXc}
\section*{\ldaTHEME}\label{spiel:theme}
\section*{\ldaDISK}\label{spiel:disk}

% NEXT CHAPTER

%\section*{\ldaTXT}
%\section*{\ldaLEX}
%	\subsection*{\ldaLEXa}
%	\subsection*{\ldaLEXb}
%\section*{\ldaGRAM}
%\section*{\ldaEXs}
%\section*{\ldaLEXc}
%\section*{\ldaTHEME}
%\section*{\ldaDISK}

\tableofcontents

\end{document}