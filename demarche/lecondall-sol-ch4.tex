\chapter{\nameref{O-ch:seinoder}}

\begin{Exercise}[title={\nameref{O-sein:gelingen}}, label={sein:gelingen}]
Traduisez les phrases suivantes:
\Question Si seulement je pouvais r\'eussir \`a l'en convaincre!
\Question Tu ne r\'eussiras jamais \`a faire assez d\'economies pour t'acheter cette voiture de luxe
\Question Comment veux tu qu'il r\'eussisse \`a r\'ealiser du profit?
\Question Le nouveau mod\`ele est assez r\'eussi
\Question Leur m\`ere a enfin r\'eussi \`a ne plus se laisser manipuler par la publicit\'e
\Question Elle n'a malheureusement pas r\'eussi sa surprise
\Question Celui qui se laisse manipuler ne peut pas r\'eussier \`a renoncer \`a consommer
\Question Je ne suis pas arriv\'e \`a mettre de l'ordre dans ses papiers
\end{Exercise}

\begin{Answer}[ref={O-sein:gelingen}]

\Question % 1. Si seulement je pouvais r\'eussir \`a l'en convaincre!

Wenn es mir nur gelungen w\"are, ihm davon zu \"uberzeugen

\Question % 2. Tu ne r\'eussiras jamais \`a faire assez d'\'economies pour t'acheter cette voiture de luxe

Es wird dir niemals gelingen, genug zu sparen, um dir dieses Luxusauto zu kaufen

\Question % 3. Comment veux-tu qu'il r\'eussisse \`a r\'ealiser du profit?

Wie stellst du dir vor, dass es ihm gelingen w\"urde, einen Gewinn zu machen.

\Question % 4. Le nouveau mod\`ele est assez r\'eussi

Das neue Modelle is erfolgreich

\Question % 5. Leur m\`ere a enfin r\'eussi \`a ne plus se laisser manipuler par la publicit\'e

Es ist ihrer Mutter endlich gelungen,
sich nicht mehr durch Werbung manipuliert zu werden

\Question % 6. Elle n'a malheureusement pas r\'eussi sa surprise

Leider ist ihr \"uberraschung nicht gelungen

\Question % 7. Celui qui se laisse manipuler ne peut pas r\'eussir \`a renoncer \`a consommer

Wem sich manipuliert l\"asst, kann es nicht gelungen, konsumieren zu verzichten

\Question % 8. Je ne suis pas arriv\'e \`a mettre de l'ordre dans ses papiers

Es ist mir nicht gelungen, Ordnung in meiner Papiere zu setzen.

\end{Answer}


\begin{Exercise}[title={\nameref{O-sein:aussen}},label={sein:aussen}]

Compl\'etez les phrases suivantes par "aussen", "\"ausserlich", "draussen", "ausser", "\"ausser...", "ausserhalb, "innen", "innerlich", "drinnen", "inner..." ou "innerhalb".

\Question Dem \ldaBLANK minister steht die Polizei zur Verf\"ugung \ldaBLANK Ordnung zu gew\"ahrleisten

\Question \ldaBLANK ist das Haus nicht ohne \"ahnlichkeit mit einem Bunker, aber man vergisst es schnell, wenn man \ldaBLANK sitzt, denn es ist sehr gem\"utlich

\Question Seine Frau will \ldaBLANK der Stadt wohnen, damit kie Kinder einen Garten zum Spielen haben

\Question Er hat mit versprochen, nicht eigensinnig zu sein und sich den \ldaBLANK Umst\"anden anzupassen

\Question Kannst du so lange \ldaBLANK warten ?

\Question Da er als \ldaBLANK minister f\"ur die \ldaBLANK Angelegenheiten nat\"urlich nicht zust\"andig ist, hat er sich geweigert, auf eine Frage \"uber die Jugendkriminalit\"at zu antworten

\Question Die \ldaBLANK stadt ist versuchsweise f\"ur den Autoverkehr gesperrt worden.

\Question \ldaBLANK ihrem Beruf haben viele Menschen keine anderen Interessen im Leben

\Question Die Gesetze gelten \"uberall \ldaBLANK der Landesgrenzen

\Question Der Artz empf\"angt keine Patienten \ldaBLANK der Sprechstundent

\end{Exercise}
\begin{comment}

\begin{Answer}[ref={O-sein:aussen}]

% Compl\'etez les phrases suivantes par "aussen", "\"ausserlich", "draussen", "ausser", "\"ausser...", "ausserhalb, "innen", "innerlich", "drinnen", "inner..." ou "innerhalb".

\Question aussen, innere

\Question Von aussen, drinnen

\Question ausserhalb

\Question aussen

\Question draussen

\Question aussen, innerlichen

\Question innere

\Question ausser

\Question innerhalb

\Question ausserhalb

\end{Answer}

\begin{Exercise}[label={sein:aussen2}]

Traduisez les phrases de l'exercice pr\'ec\'edent.

\end{Exercise}

\begin{Answer}[ref={O-sein:aussen2}]

\Question % 1. 

La police est \`a la disposition de la police pour assurer l'ordre

\Question % 2. 

De l'ext\'erieur, cette maison n'est pas sans ressemblance avec un bunker, mais on l'oublie, une fois \`a l'int\'erieur, qui est tr\`es confortable

\Question % 3. 

Sa femme veut habiter en dehors de la ville, pour que ses enfants aient un jardin dans lequel jouer

\Question % 4.

Il m'a promis de ne pas \^etre obstin\'e et de s'adapter aux circonstances ext\'erieures

\Question % 5. 

Peux tu attendre si longtemps dehors?

\Question % 6. 

Parce qu'il est ministre de l'ext\'erieur, il n'est pas responsable des affaires int\'erieures, il a refus\'e de r\'epondre \`a une question sur la criminalit\'e des jeunes.

\Question % 7. 

Le centre ville est temporairement ferm\'e \`a la circulation automobile.

\Question % 8.

En dehors de leur profession, beaucoup de gens n'ont pas d'autre int\'er\^et dans la vie

\Question % 9. 

Les lois valent \`a l'int\'erieur des fronti\`eres

\Question % 10. 

Le m\'edecin ne re�oit pas de patient en dehors des heures annonc\'ees

\end{Answer}

\begin{Exercise}[title={\nameref{O-sein:was}}, label={sein:was}]

Compl\'etez les phrases suivantes par "was" et/ou un ant\'ec\'edent logique. Attention: vous devez mettre un mot par blanc

\Question 1. \ldaBLANK er getan hat, \"uberrascht mich kaum
\Question 2.  Das ist wirklich \ldaBLANK \ldaBLANK, \ldaBLANK ich dazu sagen kann
\Question 3. Du kannst froh sein, \ldaBLANK zu haben, \ldaBLANK du brauchst 
\Question 4. Seine Papiere zu verlieren ist \ldaBLANK \ldaBLANK, \ldaBLANK einem passieren kann, wenn man im Ausland ist
\Question 5. Das ist \ldaBLANK, \ldaBLANK ich zur Zeit brauche
\Question 6. Es gibt nicht \ldaBLANK, \ldaBLANK man ihm vorwerfen kann
\Question 7. Es ist nicht \ldaBLANK Gold, \ldaBLANK gl\"anzt

\end{Exercise}
\begin{Answer}[ref={O-sein:was}]

\Question % 1. \ldaBLANK er getan hat, \"uberrascht mich kaum
Was
\Question % 2.  Das ist wirklich \ldaBLANK \ldaBLANK, \ldaBLANK ich dazu sagen kann
nicht, alles, was
\Question % 3. Du kannst froh sein, \ldaBLANK zu haben, \ldaBLANK du brauchst 
alles, was
\Question % 4. Seine Papiere zu verlieren ist \ldaBLANK \ldaBLANK, \ldaBLANK einem passieren kann, wenn man im Ausland ist
leider, wirklich, was
\Question % 5. Das ist \ldaBLANK, \ldaBLANK ich zur Zeit brauche
alls, was
\Question % 6. Es gibt nicht \ldaBLANK, \ldaBLANK man ihm vorwerfen kann
viel, das
\Question % 7. Es ist nicht \ldaBLANK Gold, \ldaBLANK gl\"anzt
alles, was

\end{Answer}

\begin{Answer}[ref={spiel:theme}]

\begin{Exercise}[title={\ldaTHEME}, label={sein:theme}]

\Question Si les gens n'achetaient que ce dont ils ont vraiment besoin, il y aurait beaucoup moins de gaspillage

\Question Quand le march\'e est satur\'e, la production exc\`ede tr\`es vite les besoins

\Question Pour lancer ce nouveau produit qui ne correspond pas \`a un besoin, cette entreprise a d� s'adresser \`a une agence publicitaire

\Question Dans le prix de vente de beaucoup de produits, la part des frais de publicit\'e et de l'emballage est souvent plus \'elev\'e que celle de la fabrication proprement dite

\Question Mon oncle d'Am\'erique m'a assur\'e qu'il changeait tous les ans de voiture pour ne pas se faire remarquer

\Question Le consommateur moyen ne se doute pas que ses besoins sont cr\'e\'es par la publicit\'e sans laquelle la soci\'et\'e moderne est inconcevable

\Question Au lieu de faire r\'eparer son viel appareil de t\'el\'evision, il pr\'ef\`ere en acheter un neuf \`a cr\'edit

\Question Sachant qu'il a tout \`a profusion chez lui, je ne sais pas quoi lui offrir \`a son anniversaire

\Question En p\'eriode de crise, les agens de publicit\'e se frottent les mains

\Question Ses coll\`egues de travail l'\'evitent, car il s'est rendu suspect en venant sa voiture pour limiter ses d\'epenses

\Question Le pauvre: apr\`es avoir \'et\'e expuls\'e de sons propre pays parce qu'il vantait sans cesse les avantages de la soci\'et\'e de consommation occidentale, il vit maintenant de l'aide sociale en RFA et il ne peut rien acheter hormis les choses d'usage courant.

\Question Je ne pourrai jamais me pardonner de ne m'�tre pa tu

\Question Apr\`es avoir vendu son magasin de fournitures de bureau, sa femme a ouvert une boutique de mode

\Question Il est soup�onn\'e d'appartenir \`a une organisation terroriste parce qu'il a occup\'e une maison vide avec quelques autres \'etudiants en sociologie.

\end{Exercise}


\begin{comment}

\begin{Answer}[ref={O-sein:theme}]

\Question % 1. Si les gens n'achetaient que ce dont ils ont vraiment besoin, il y aurait beaucoup moins de gaspillage

Wenn die Leute kauften was sie nur wirklich brauchten, w\"urde viel weniger verschwendet.

\Question % 2. Quand le march\'e est satur\'e, la production exc\`ede tr\`es vite les besoins

Wenn der Markt nicht ges\"attigt ist, \"ubersteigt die Produktion sehr schnell den Bedarf

\Question % 3. Pour lancer ce nouveau produit qui ne correspond pas \`a un besoin, cette entreprise a d� s'adresser \`a une agence publicitaire

Un dieses neue Produkt auf den Markt zu bringen, d\"urfte diese Firma eine Werbeagentur beauftragen.

\Question % 4. Dans le prix de vente de beaucoup de produits, la part des frais de publicit\'e et de l'emballage est souvent plus \'elev\'e que celle de la fabrication proprement dite

In der Preis vieler G\"uter, ist der Teil der Werbung und Verpackung oft gr\"osser als derjenige der Herstellung.

\Question % 5. Mon oncle d'Am\'erique m'a assur\'e qu'il changeait tous les ans de voiture pour ne pas se faire remarquer

Mein Unkel aus Amerika hat mir versichert, dass er jedes Jahres sein Auto w\"achselt, um nicht auffallen zu werden% [Text: Wer in Amerika keinen Fernsehapparat besitzt, f\"allt auf]

\Question % 6. Le consommateur moyen ne se doute pas que ses besoins sont cr\'e\'es par la publicit\'e sans laquelle la soci\'et\'e moderne est inconcevable

Der Konsument verzweifelt sich nicht, dass seine Bed\"urfnisse von der Werbung geweckt sind, ohne derer ist die moderne Gesellschaft undenkbar.  

\Question % 7. Au lieu de faire r\'eparer son viel appareil de t\'el\'evision, il pr\'ef\`ere en acheter un neuf \`a cr\'edit

Anstatt seinen alten Fernsehapparat zu reparieren, will er lieber einen neuen kaufen.

\Question % 8. Sachant qu'il a tout \`a profusion chez lui, je ne sais pas quoi lui offrir \`a son anniversaire

Wissend, dass er zu Hause alles in H\"ulle un F\"ulle hat, ich weiss nicht, was ihm zu seiner Geburtstag zu anbieten.

\Question % 9. En p\'eriode de crise, les agens de publicit\'e se frottent les mains

In Zeiten der Krise, reiben sie sich die H\"ande

\Question % 10. Ses coll\`egues de travail l'\'evitent, car il s'est rendu suspect en venant sa voiture pour limiter ses d\'epenses

Seine Berufskollegen meiden ihn, weil er sich verd\"achtig, indem er sein Auto verkauft hat, um seine Ausgabe abzunehmen 

\Question % 11. Le pauvre: apr\`es avoir \'et\'e expuls\'e de sons propre pays parce qu'il vantait sans cesse les avantages de la soci\'et\'e de consommation occidentale, il vit maintenant de l'aide sociale en RFA et il ne peut rien acheter hormis les choses d'usage courant.

Der Ungl\"ucklicher, nachdem er aus seinem selben Land ausgewiesen wurde, denn er immer die Vorz\"uge der Westlichen Konsumgesellschaft pries, lebt er jetzt aus Sozialhilfe des BRD, und kann sich nichts leider ausser von t\"aglichen Bedarf. 

\Question % 12. Je ne pourrai jamais me pardonner de ne m'�tre pa tu

Ich werde mir niemals verzweifeln, mich nicht schweigen zu haben.

\Question % 13. Apr\`es avoir vendu son magasin de fournitures de bureau, sa femme a ouvert une boutique de mode

Nachdem er sein B\"urobedarf Gesch\"aft verkauft hat, hat seine Frau ein Modegesch\"aft ge\"offnet.

\Question % 14. Il est soup�onn\'e d'appartenir \`a une organisation terroriste parce qu'il a occup\'e une maison vide avec quelques autres \'etudiants en sociologie.

Er steht im Verdacht, zu einer terroristischen Organisation zu geh\"oren, denn er  mit einigen Sociologie Studenten in einem leeren Haus gewohnt hat.

\end{Answer}

\end{comment}

%\begin{Exercise}[
%	title={\ldaDISK}, 
%label={sein:disk}]

%\Question % Ist der Verbraucher ein Spielzeug der Industrie?
%\Question % Der Komsumverzicht: Utopie oder Zivilcourage?
%\Question % Soll Werbung nur informieren oder auch W\"unsche wecken?
%\Question % Das heutige Konsumverhalten: ein ausgleich f\"ur die verlorene Regiosit\"at von fr\"uher?
%\end{Exercise}

\begin{comment}
\end{comment}